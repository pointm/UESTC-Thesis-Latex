\documentclass[master]{thesis-uestc}
% 此处-----------|---的模板类型设置请参照README
\title{大功率宽频带脊波导微波窗研究}{English Title} % 论文题目
\author{方源}{English Name} % 作者姓名
\setdate[submit]{ } % 论文提交日期,可留空
\setdate[oral]{ }  % 答辩日期,可留空
\setdate[confer]{ } % 学位授予日期,可留空
\advisor{\qquad 王建勋\qquad\qquad 教\chinesespace 授}{English name English title}
% \coAdvisor{合作导师姓名\chinesespace 导师职称}{Co advisor English name English title} % 仅专业硕士/博士使用,在扉页/英文首页添加合作导师,不使用请注释
\school{电子科学与工程学院}{School of Electronic Science and Engineering} % 学院信息
\major{电子科学与技术}{Electronic Science and Technology} % 专业信息
\studentnumber{202221020122} % 学号
\ProfessionalDegreeArea{随便学学} % 专业硕士专用:专业学位领域
\ClassificationNumber{TP309.2} % 分类号
\ClassifiedClass{公开} % 密级
\UDCNumber{004.78} % UDC号
\Chairman{xxxxx} % 答辩委员会主席

% 取消注释以下内容,用于禁止文中换行处的英语单词自动截断换行。
% \tolerance=1
% \emergencystretch=\maxdimen
% \hyphenpenalty=10000
% \hbadness=10000

\makeglossaries % 产生缩略词表/符号表专用,不使用时请注释。 注意,之后的acronym,glossaryentry以及相关的引用也请注释。
\newacronym[description=逻辑卷管理器]{lvm}{LVM}{Logical Volume Manager} % 定义缩略词:以本项为例,逻辑卷管理器为中文名称;lvm用于文内引用;LVM为显示的应为缩略语或符号;Logical Volume Manager为显示的英文全称/描述。
\newglossaryentry{tree}{name={tree}, description={trees are the better humans}}  % 定义符号:以本项为例,name={tree}为符号名称;tree用于文内引用; description={trees are the better humans}为显示的描述。页码自动添加。

\begin{document}

\makecover % 封面+中英文扉页
\originalitydeclaration % 原创新声明
% \signatureofdeclaration{signature.pdf} % 用于添加扫描版签字后的原创新声明(使用时取消注释本行,并注释掉上一行)
% 中文摘要
\begin{chineseabstract}

    \chinesekeyword{xxx,xxx,xxx} % 中文关键词
\end{chineseabstract}
% 英文摘要
\begin{englishabstract}

    \englishkeyword{xxx, xxx, xxx} % 英文关键词
\end{englishabstract}

\thesistableofcontents % 目录
\thesisfigurelist % 图目录,仅在需要时添加,一般情况下请注释
\thesistablelist % 表目录,仅在需要时添加,一般情况下请注释
% \glsaddall % 默认仅显示被正文引用的项,取消注释以显示所有已定义的缩略词/符号
\thesisglossarylist % 缩略词表,仅在需要时添加,一般情况下请注释
\thesissymbollist % 符号表,仅在需要时添加,一般情况下请注释

% 正文内容


\chapter{绪\hspace{6pt}论}

\chapter{微波窗的理论研究}
\section{经典微波窗的理论分析}
\subsection{盒型窗的理论研究}
\subsection{同轴窗的理论研究}
\section{阻抗匹配的设计}
\subsection{常见微波传输线的特性阻抗}
\subsection{脊波导的阻抗匹配设计}
测试一下相关的引用\citing{114514}
\chapter{6-18GHz 宽频带脊波导窗的设计}
\section{传统盒型窗的局限性}
\section{输入窗的相关研究}
\subsection{计算脊波导窗功率容量}
\subsubsection{使用最大场强法计算脊波导窗的功率容量}
\subsubsection{使用二次电子发射计算脊波导窗的功率容量}
\subsection{关于窗的多物理场分析}
\subsubsection{窗片的温度场分析}
\subsubsection{窗片的应力分析}
\section{模式变换部分的设计}
\section{输出窗的相关研究}

\subsection{计算脊波导窗功率容量}
\subsubsection{使用最大场强法计算脊波导窗的功率容量}
\subsubsection{使用二次电子发射计算脊波导窗的功率容量}
\subsection{关于窗的多物理场分析}
\subsubsection{窗片的温度场分析}
\subsubsection{窗片的应力分析}
\section{小结}
\chapter{L 波段宽频带脊波导窗的设计}
\section{频带搬移时候出现的问题分析}

\section{L 波段宽频带脊波导窗的结构设计}

\section{计算L波段脊波导窗功率容量}
\subsection{使用最大场强法计算脊波导窗的功率容量}
\subsection{使用二次电子发射计算脊波导窗的功率容量}
\section{关于L波段脊波导窗片的多物理场分析}
\subsection{窗片的温度场分析}
\subsection{窗片的应力分析}
\section{模式变换部分的设计}
\section{小结}
\chapter{窗片测试产生的误差分析}
\chapter{全文总结与展望}
\chapter{README图表与相关引用规范}

角标参考文献\citing{chen2001hao}测试,普通参考文献~\cite{clerc2010discrete}。

这是符号\gls{tree}\cite{liuxf2006}。

$\hat{H}, f(x)$, $\vec{V}$

$$\hat{H}$$

$\mathcal{C}_i$

这是缩略词\acrlong{lvm}的长引用,这是缩略词的短引用\acrshort{lvm}。

\begin{figure}[!htb]
    \includegraphics[width=0.5\linewidth]{pic/figure.pdf}
    \caption[short catption 1]{Test caption 1}
\end{figure}


\begin{figure}[!htb]
    \small
    \centering
    \begin{tabular}{@{\ }c@{\ }c}
        \includegraphics[width=0.49\textwidth]{pic/figure.pdf} & 
        \hspace{5pt}
        \includegraphics[width=0.49\textwidth]{pic/figure.pdf}     \\
        \mbox{\small (a)随便试试的超级长的标题}                                                                               & 
        \mbox{\small (b)随便试试的超级长的标题}                                                                                  \\
    \end{tabular}
    \caption{随便试试的超级长的标题-总}
    \label{fig:test}
\end{figure}

\begin{figure}[!htbp]
    \centering
    \begin{subfigure}[t]{0.35\linewidth}
        \centering
        \includegraphics[scale=0.25]{logo.pdf}
        \caption{Fig. 1}
        \label{fig:1-1}
    \end{subfigure}
    \begin{subfigure}[t]{0.35\linewidth}
        \centering
        \includegraphics[scale=0.25]{logo.pdf}
        \caption{Fig. 2}
        \label{fig:1-2}
    \end{subfigure}
    \\[6bp]
    \begin{subfigure}[t]{0.35\linewidth}
        \centering
        \includegraphics[scale=0.25]{logo.pdf}
        \caption{Fig. 3}
        \label{fig:1-3}
    \end{subfigure}
    \begin{subfigure}[t]{0.35\linewidth}
        \centering
        \includegraphics[scale=0.25]{logo.pdf}
        \caption{Fig. 4}
        \label{fig:1-4}
    \end{subfigure}
    \caption{Fig.}
    \label{fig:1}
\end{figure}
算法框:

\begin{algorithm}[H]\label{alg:1}
    \KwData{this text}
    \KwResult{how to write algorithm with \LaTeX2e}
    initialization\;
    \While{not at end of this document}{
        read current\;
        \eIf{understand}{
            go to next section\;
            current section becomes this one\;
        }{
            go back to the beginning of current section\;
        }
    }
    \caption{How to wirte an algorithm.}
\end{algorithm}

这是算法\algref{alg:1}。

\thesisacknowledgement

xxxx % 直接填写致谢内容,写法与正文一致

\thesisbibliography[large]{reference} % 参考文献

\thesisappendix
\chapter{xxxx} % 直接填写附录内容,写法与正文一致

% 攻读学位期间成果(本科不添加),例如:
\begin{thesistheaccomplish}
    \section{学术论文}
    \bibitem{SGXDedup} \textbf{Ren, Yanjing} and Li, Jingwei and Yang, Zuoru and Lee, Patrick PC and Zhang, Xiaosong. Accelerating Encrypted Deduplication via SGX[C]. Proc.of USENIX ATC, 2021, 957-971. \textbf{CCF-A}
    \section{发明专利}
    \bibitem{CN111338572B} 李经纬, 杨祚儒, \textbf{任彦璟}, 李柏晴, 张小松. 一种可调节加密重复数据删除方法:CN111338572B[P]. 2021-09-14.
\end{thesistheaccomplish}

% 以下为本科同学专用,插入文献翻译
\thesistranslationoriginal
\section{Tahoe-LAFS: The Least-Authority File System}
% \insertPDFPage{} % 用于插入单页PDF文件,例如原始文献(该操作会导致页码被取消,请谨慎使用)

\thesistranslationchinese
\section{Tahoe-LAFS:最小权限文件系统}

\end{document}
