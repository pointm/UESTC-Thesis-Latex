\documentclass[master]{thesis-uestc}
% 此处-----------|---的模板类型设置请参照README
\title{大功率宽频带脊波导微波窗研究}{English Title} % 论文题目
\author{方源}{English Name} % 作者姓名
\setdate[submit]{ } % 论文提交日期,可留空
\setdate[oral]{ }  % 答辩日期,可留空
\setdate[confer]{ } % 学位授予日期,可留空
\advisor{\qquad 王建勋\qquad\qquad 教\chinesespace 授}{English name English title}
% \coAdvisor{合作导师姓名\chinesespace 导师职称}{Co advisor English name English title} % 仅专业硕士/博士使用,在扉页/英文首页添加合作导师,不使用请注释
\school{电子科学与工程学院}{School of Electronic Science and Engineering} % 学院信息
\major{电子科学与技术}{Electronic Science and Technology} % 专业信息
\studentnumber{202221020122} % 学号
\ProfessionalDegreeArea{随便学学} % 专业硕士专用:专业学位领域
\ClassificationNumber{TP309.2} % 分类号
\ClassifiedClass{公开} % 密级
\UDCNumber{004.78} % UDC号
\Chairman{xxxxx} % 答辩委员会主席

% 取消注释以下内容,用于禁止文中换行处的英语单词自动截断换行。
% \tolerance=1
% \emergencystretch=\maxdimen
% \hyphenpenalty=10000
% \hbadness=10000

\makeglossaries % 产生缩略词表/符号表专用,不使用时请注释。 注意,之后的acronym,glossaryentry以及相关的引用也请注释。
\newacronym[description=逻辑卷管理器]{lvm}{LVM}{Logical Volume Manager} % 定义缩略词:以本项为例,逻辑卷管理器为中文名称;lvm用于文内引用;LVM为显示的应为缩略语或符号;Logical Volume Manager为显示的英文全称/描述。
\newglossaryentry{tree}{name={tree}, description={trees are the better humans}}  % 定义符号:以本项为例,name={tree}为符号名称;tree用于文内引用; description={trees are the better humans}为显示的描述。页码自动添加。

\begin{document}

\makecover % 封面+中英文扉页
\originalitydeclaration % 原创新声明
% \signatureofdeclaration{signature.pdf} % 用于添加扫描版签字后的原创新声明(使用时取消注释本行,并注释掉上一行)
% 中文摘要
\begin{chineseabstract}

    \chinesekeyword{xxx,xxx,xxx} % 中文关键词
\end{chineseabstract}
% 英文摘要
\begin{englishabstract}

    \englishkeyword{xxx, xxx, xxx} % 英文关键词
\end{englishabstract}

\thesistableofcontents % 目录
\thesisfigurelist % 图目录,仅在需要时添加,一般情况下请注释
\thesistablelist % 表目录,仅在需要时添加,一般情况下请注释
% \glsaddall % 默认仅显示被正文引用的项,取消注释以显示所有已定义的缩略词/符号
\thesisglossarylist % 缩略词表,仅在需要时添加,一般情况下请注释
\thesissymbollist % 符号表,仅在需要时添加,一般情况下请注释

% 正文内容


\chapter{绪\hspace{6pt}论}
\section{生活中的微波真空管器件}
\section{真空管的微波窗}
\section{真空管中的宽带宽需求}
\section{国内外研究现状}
\section{常见的几种微波窗宽带宽方案}
现阶段常用的几种宽带宽微波窗的实现方案可以被总结为如下几点,这些方案主要用于改进微波窗与输能波导之间的匹配部分。有的方案中,将圆形窗片与标准波导之间做了渐变段变换,使得窗片的匹配带宽得到增宽\citing{cook_broadband_2013_gradually}。这种
\section{论文结构与安排}
\chapter{微波窗的理论研究}
\section{经典微波窗的理论分析}
随着时代的发展,微波窗的这种累也趋于多种多样,在理论研究方面比较完备的微波窗是盒型窗与同轴窗,这两种窗的结构较为简单便于使用场匹配或者等效电路法进行研究。由于对这两种窗的结构理论研究有助于后续的窗片设计,故现在此对这两种窗的基本理论进行相应的论述。
\subsection{盒型窗的理论研究}
常见的盒型窗的主要是由负责进行传输的矩形波导、一个薄圆片形状的介质窗片以及窗片两侧的匹配圆柱波导组成,这其中蕴含了的电磁场边界条件的不连续性,可以使用等效电路法来进行研究。此时将盒型窗的两侧波导以及窗片两侧的空气圆柱部分等效为波导传输线,将较为薄的窗片、矩形波导与圆柱形波导之间的不连续性和窗片与圆柱形波导之间的不连续性等效为电纳。并以此等效思路为基础构建微波窗的等效电路,进而求解电路的频率响应。

常规的盒型窗的矩形波导的工作频率为$TE_{10}$模式,其中的圆形窗片的主要工作模式为$TE_{11}$模式,这两种模式均为波导的基模。当矩形波导和圆波导进行直接耦合的时候根据文章\citing{Bharathi2015DESIGNAD, marcuvitz_waveguide_1986_rec_cylinder}的研究表明,此时会出现不连续性,在等效电路中相当于引入了一个等效电纳$B_{T}$,此时根据式子\ref{eq:B_T}这个等效电纳可以被表示为:
\begin{equation}\label{eq:B_T}
    B_{T}=\frac{b}{\lambda_{g r}}\left\{2 \ln \left(\frac{D^{2}-b^{2}}{4 b D}\right)+\left(\frac{b}{D}+\frac{D}{b}\right) \ln \left(\frac{D+b}{D-b}\right)+2 \sum_{n=1}^{\infty} \frac{\sin ^{2} n \phi}{n^{3} \phi^{2}} \delta_{2 n}\right\}
\end{equation}
此时式子中的$D$为圆形波导的直径,$a$和$b$分别为矩形波导的长边长度和短边长度,$\beta$为矩形波导的传播常数,$\lambda_{g r}$为矩形波导的工作波长,$\lambda_{g c}$为圆形波导的工作波长,$t$为圆柱形介质窗片的厚度,$\omega$为微波窗的角工作频率,$c$为真空的光速,$\epsilon_{r}$为矩形窗片介质片的相对介电常数,$\lambda$为自由空间中的波长。

为了便于封装焊接,窗片的直径最好等于矩形波导的对角线长度,此时根据勾股定理我们可以得到窗片直径$D$和矩形波导长短边之间的关系$D=\sqrt{a^2+b^2} $。关于计算式最后的求和式中的$\delta_{2n}$和$\phi$的计算方法可以被表示为式子\ref{eq:delta_phi}中的\ref{eq:delta}到\ref{eq:phi}所示:
\begin{subequations}\label{eq:delta_phi}
\begin{align}
    \delta_{2 n} &= \frac{1}{\sqrt{1-\left(\frac{\beta \mathrm{D}}{2 \pi n}\right)^{2}}}-1 \label{eq:delta}\\
    \beta &= \frac{2 \pi}{\lambda_{gr}} \label{eq:beta}\\
    \phi &= \frac{\pi b}{D} \label{eq:phi}
\end{align}
\end{subequations}
根据之前分析的盒型窗中存在的三个不连续性边界条件,可以将原先的盒型窗等效为如--所示的等效电路,在这个等效电路中,$Z_{1}$是真空填充的圆柱形波导的特征阻抗,$Z_{2}$是矩形波导的特征阻抗,$B_{d}$是圆形窗片引入不连续性所等效的归一化电纳,$B_{T}$是由于矩形波导与圆柱形波导之间的不连续性引入的归一化电纳。

根据微波网络理论,这个两端口的等效电路可以被表示为一个如\ref{eq:ABCD}所示的$ABCD$矩阵,其中$k=Z_{1}/Z_{2}$,是真空填充的圆柱形的特征阻抗按照矩形波导的特征阻抗所归一化的特征阻抗比值;$\gamma=\frac{2 \pi}{\lambda_{gr}}$是真空填充的圆柱形波导的传播常数,$l$是介质窗片两侧的真空填充的圆柱形波导的纵向长度。
\begin{equation}\label{eq:ABCD}
    \begin{split}
        \begin{bmatrix}
            A & B \\
            C & D
        \end{bmatrix} 
        & = 
        \begin{bmatrix}
            \sqrt{k} & 0 \\
            jB_{T}\sqrt{k} & 1/\sqrt{k}
        \end{bmatrix}
        \begin{bmatrix}
            \cos{\gamma l} & j\sin{\gamma l} \\
            j\sin{\gamma l} & \cos{\gamma l}
        \end{bmatrix}
        \begin{bmatrix}
            1 & 0 \\
            jB_{d} & 1
        \end{bmatrix} \\
        & \quad \times 
        \begin{bmatrix}
            \cos{\gamma l} & j\sin{\gamma l} \\
            j\sin{\gamma l} & \cos{\gamma l}
        \end{bmatrix}
        \begin{bmatrix}
            1/\sqrt{k} & 0 \\
            jB_{T}\sqrt{k} & \sqrt{k}
        \end{bmatrix}
    \end{split}
\end{equation}

由于此时的窗片的厚度相比于使用窗片介质填充的圆柱形波导的工作波长来说太短,其引入的不连续性可以被等效为电纳$B_{d}$,详细的计算式是式\ref{eq:B_d},其中$t$为窗片的厚度,$\varepsilon_d$为窗片的相对介电常数:
\begin{equation}\label{eq:B_d}
    B_{d} = t (\varepsilon_d - 1) \left( \frac{\omega}{c} \right) \left( \frac{\lambda_{gc}}{\lambda} \right)
\end{equation}

此时,假设盒型窗的输入功率为$P_{1}$,通过盒型窗之后输出功率为$P_{2}$,根据S参数以及$ABCD$矩阵的定义,我们可以得到$P_{2}$与$P_{1}$的比值为式子\ref{eq:P2P1}:
\begin{equation}\label{eq:P2P1}
    \frac{P_{2}}{P_{1}} = \frac{1}{1+\frac{1}{4} (B-C)^2}
\end{equation}
那么根据能量守恒定律,假设窗片不存在损耗,那么其反射系数$|\Gamma|$可以被表示为式子\ref{eq:Gamma}:
\begin{equation}\label{eq:Gamma}
    |\Gamma| = \left| 1 - \left( \frac{P_{2}}{P_{1}} \right) \right|^{\frac{1}{2}}
\end{equation}
假设窗片完全传输,那么此时的$P_{2}=P_{1}$,进而根据式子\ref{eq:P2P1}可以推理出来$ABCD$矩阵中的$B=C$。为了获得一个可以确定盒型窗的几何参数的进行求解的方程,我们假设此时的圆柱形波导长度$l$和要求匹配到的频点$f$均为已经确定的值。

现在我们将原先式子\ref{eq:ABCD}的几个矩阵元素完全相乘,可以得到计算式\ref{eq:ABCD_solved},式子中\ref{eq:A}到\ref{eq:D}为计算式\ref{eq:ABCD}中$ABCD$矩阵的四个元素:
\begin{subequations}\label{eq:ABCD_solved}
    \begin{align}
        A &= \frac{1}{2} \big[-(B_d + 2 B_T k) \sin(2\gamma l) + (2 - B_d B_T k)\cos(2\gamma l) + B_d B_T k \big] \label{eq:A} \\
        B &= -i k \sin(\gamma l) (B_d \sin(\gamma l) - 2 \cos(\gamma l)) \label{eq:B} \\
        C &= \frac{i (\cos(\gamma l) - B_T k \sin(\gamma l)) \big[(2 - B_d B_T k)\sin(\gamma l) + (B_d + 2 B_T k)\cos(\gamma l)\big]}{k} \label{eq:C} \\
        D &= \frac{1}{2} \big[-(B_d + 2 B_T k) \sin(2\gamma l) + (2 - B_d B_T k)\cos(2\gamma l) + B_d B_T k \big] \label{eq:D}
    \end{align}
\end{subequations}

注意到推演结果中$A$元素与$D$元素相等,这与微波网络中的理论描述相符。此时为了满足$ABCD$矩阵的$B$与$C$元素相等,我们可以将这两个元素进行作差处理,如果相等的话那么可以得到$B-C=0$,进一步的,可以得到如下关于作差后的表达式\ref{eq:B-C}:

\begin{equation}\label{eq:B-C}
    \begin{split}
        B - C = & -\frac{i}{2k} \Big[ \big(-B_{d}(B_{T}^2 + 1)k^2 + B_{d} + 4B_{T}k\big)\cos(2\gamma l) \\
                & + B_{d}(B_{T}^2 + 1)k^2 - 2\big[k(B_{T}(B_{d} + B_{T}k) + k) - 1\big]\sin(2\gamma l) \\
                & + B_{d} \Big] = 0
    \end{split}
\end{equation}
\subsection{同轴窗的理论研究}
\section{阻抗匹配的设计}
\subsection{常见微波传输线的特性阻抗}

\subsection{双脊波导的特性与相关参数}

\subsection{双脊波导的匹配设计}
脊波导由于其主模带宽非常宽,其经常由于优秀的几何结构特性便于被设计为过渡波导。并且由于其在真空填充时特性阻抗位于常用的标准矩形波导(377$\Omega$ )与同轴线(50$\Omega$)之间便于被设计为矩形波导与同轴线之间的过渡波导,故其是在许多宽带微波器件中常用的微波传输线之一\citing{helszajn_ridge_2000}。并且由于同频率之下的脊波导的长边长度要短于矩形波导,更加有利于器件的小型化。为了便于后续窗片的测量,有必要确定脊波导到标准矩形波导之间的过渡段的设计方案。常见的脊波导到矩形波导的过渡
\chapter{6-18GHz 宽频带脊波导窗的设计}
\section{传统盒型窗的局限性}
\section{输入窗的相关研究}
\subsection{计算脊波导窗功率容量}
\subsubsection{使用最大场强法计算脊波导窗的功率容量}
\subsubsection{使用二次电子发射计算脊波导窗的功率容量}
\subsection{关于窗的多物理场分析}
\subsubsection{窗片的温度场分析}
\subsubsection{窗片的应力分析}
\section{模式变换部分的设计}
\section{输出窗的相关研究}

\subsection{计算脊波导窗功率容量}
\subsubsection{使用最大场强法计算脊波导窗的功率容量}
\subsubsection{使用二次电子发射计算脊波导窗的功率容量}
\subsection{关于窗的多物理场分析}
\subsubsection{窗片的温度场分析}
\subsubsection{窗片的应力分析}
\section{小结}
\chapter{L 波段宽频带脊波导窗的设计}
\section{频带搬移时候出现的问题分析}

\section{L 波段宽频带脊波导窗的结构设计}

\section{计算L波段脊波导窗功率容量}
\subsection{使用最大场强法计算脊波导窗的功率容量}
\subsection{使用二次电子发射计算脊波导窗的功率容量}
\section{关于L波段脊波导窗片的多物理场分析}
\subsection{窗片的温度场分析}
\subsection{窗片的应力分析}
\section{模式变换部分的设计}
\section{小结}
\chapter{窗片测试产生的误差分析}
\chapter{全文总结与展望}
\chapter{README图表与相关引用规范}

角标参考文献\citing{chen2001hao}测试,普通参考文献~\cite{clerc2010discrete}。

这是符号\gls{tree}\cite{liuxf2006}。

$\hat{H}, f(x)$, $\vec{V}$

$$\hat{H}$$

$\mathcal{C}_i$

这是缩略词\acrlong{lvm}的长引用,这是缩略词的短引用\acrshort{lvm}。

\begin{figure}[!htb]
    \includegraphics[width=0.5\linewidth]{pic/figure.pdf}
    \caption[short catption 1]{Test caption 1}
\end{figure}


\begin{figure}[!htb]
    \small
    \centering
    \begin{tabular}{@{\ }c@{\ }c}
        \includegraphics[width=0.49\textwidth]{pic/figure.pdf} & 
        \hspace{5pt}
        \includegraphics[width=0.49\textwidth]{pic/figure.pdf}     \\
        \mbox{\small (a)随便试试的超级长的标题}                                                                               & 
        \mbox{\small (b)随便试试的超级长的标题}                                                                                  \\
    \end{tabular}
    \caption{随便试试的超级长的标题-总}
    \label{fig:test}
\end{figure}

\begin{figure}[!htbp]
    \centering
    \begin{subfigure}[t]{0.35\linewidth}
        \centering
        \includegraphics[scale=0.25]{logo.pdf}
        \caption{Fig. 1}
        \label{fig:1-1}
    \end{subfigure}
    \begin{subfigure}[t]{0.35\linewidth}
        \centering
        \includegraphics[scale=0.25]{logo.pdf}
        \caption{Fig. 2}
        \label{fig:1-2}
    \end{subfigure}
    \\[6bp]
    \begin{subfigure}[t]{0.35\linewidth}
        \centering
        \includegraphics[scale=0.25]{logo.pdf}
        \caption{Fig. 3}
        \label{fig:1-3}
    \end{subfigure}
    \begin{subfigure}[t]{0.35\linewidth}
        \centering
        \includegraphics[scale=0.25]{logo.pdf}
        \caption{Fig. 4}
        \label{fig:1-4}
    \end{subfigure}
    \caption{Fig.}
    \label{fig:1}
\end{figure}
算法框:

\begin{algorithm}[H]\label{alg:1}
    \KwData{this text}
    \KwResult{how to write algorithm with \LaTeX2e}
    initialization\;
    \While{not at end of this document}{
        read current\;
        \eIf{understand}{
            go to next section\;
            current section becomes this one\;
        }{
            go back to the beginning of current section\;
        }
    }
    \caption{How to wirte an algorithm.}
\end{algorithm}

这是算法\algref{alg:1}。

\thesisacknowledgement

xxxx % 直接填写致谢内容,写法与正文一致

\thesisbibliography[large]{reference.bib} % 参考文献

\thesisappendix
\chapter{xxxx} % 直接填写附录内容,写法与正文一致

% 攻读学位期间成果(本科不添加),例如:
\begin{thesistheaccomplish}
    \section{学术论文}
    \bibitem{SGXDedup} \textbf{Ren, Yanjing} and Li, Jingwei and Yang, Zuoru and Lee, Patrick PC and Zhang, Xiaosong. Accelerating Encrypted Deduplication via SGX[C]. Proc.of USENIX ATC, 2021, 957-971. \textbf{CCF-A}
    \section{发明专利}
    \bibitem{CN111338572B} 李经纬, 杨祚儒, \textbf{任彦璟}, 李柏晴, 张小松. 一种可调节加密重复数据删除方法:CN111338572B[P]. 2021-09-14.
\end{thesistheaccomplish}

% 以下为本科同学专用,插入文献翻译
\thesistranslationoriginal
\section{Tahoe-LAFS: The Least-Authority File System}
% \insertPDFPage{} % 用于插入单页PDF文件,例如原始文献(该操作会导致页码被取消,请谨慎使用)

\thesistranslationchinese
\section{Tahoe-LAFS:最小权限文件系统}

\end{document}
